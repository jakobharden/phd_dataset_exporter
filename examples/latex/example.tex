% Minimal working application example
%
% Copyright 2023 Jakob Harden (jakob.harden@tugraz.at, Graz University of Technology, Graz, Austria)
% License: MIT
% This file is part of the PhD thesis of Jakob Harden.
% 
% Permission is hereby granted, free of charge, to any person obtaining a copy of this software and associated 
% documentation files (the “Software”), to deal in the Software without restriction, including without 
% limitation the rights to use, copy, modify, merge, publish, distribute, sublicense, and/or sell copies of 
% the Software, and to permit persons to whom the Software is furnished to do so, subject to the following conditions:
% 
% THE SOFTWARE IS PROVIDED “AS IS”, WITHOUT WARRANTY OF ANY KIND, EXPRESS OR IMPLIED, INCLUDING BUT NOT LIMITED TO 
% THE WARRANTIES OF MERCHANTABILITY, FITNESS FOR A PARTICULAR PURPOSE AND NONINFRINGEMENT. IN NO EVENT SHALL THE 
% AUTHORS OR COPYRIGHT HOLDERS BE LIABLE FOR ANY CLAIM, DAMAGES OR OTHER LIABILITY, WHETHER IN AN ACTION OF CONTRACT, 
% TORT OR OTHERWISE, ARISING FROM, OUT OF OR IN CONNECTION WITH THE SOFTWARE OR THE USE OR OTHER DEALINGS IN THE SOFTWARE.
%
% preamble, set document class and import packages
\documentclass[12pt,a4paper]{article}
\usepackage[utf8]{inputenc}
\usepackage[T1]{fontenc}
\usepackage[english]{babel}
\usepackage{amsmath}
\usepackage{amsfonts}
\usepackage{amssymb}
\usepackage{graphicx}
\usepackage[backend=bibtex,style=numeric,sorting=none]{biblatex}
\addbibresource{example.bib}

% import additional commands (see also: oct2texdefs.tex)
% TeX commands to conveniently use serialized dataset content
%
% Copyright 2023 Jakob Harden (jakob.harden@tugraz.at, Graz University of Technology, Graz, Austria)
% License: MIT
% This file is part of the PhD thesis of Jakob Harden.
% 
% Permission is hereby granted, free of charge, to any person obtaining a copy of this software and associated 
% documentation files (the “Software”), to deal in the Software without restriction, including without 
% limitation the rights to use, copy, modify, merge, publish, distribute, sublicense, and/or sell copies of 
% the Software, and to permit persons to whom the Software is furnished to do so, subject to the following conditions:
% 
% THE SOFTWARE IS PROVIDED “AS IS”, WITHOUT WARRANTY OF ANY KIND, EXPRESS OR IMPLIED, INCLUDING BUT NOT LIMITED TO 
% THE WARRANTIES OF MERCHANTABILITY, FITNESS FOR A PARTICULAR PURPOSE AND NONINFRINGEMENT. IN NO EVENT SHALL THE 
% AUTHORS OR COPYRIGHT HOLDERS BE LIABLE FOR ANY CLAIM, DAMAGES OR OTHER LIABILITY, WHETHER IN AN ACTION OF CONTRACT, 
% TORT OR OTHERWISE, ARISING FROM, OUT OF OR IN CONNECTION WITH THE SOFTWARE OR THE USE OR OTHER DEALINGS IN THE SOFTWARE.
%
%-------------------------------------------------------------------------------
% Load etoolbox and other required pgf packages
\usepackage{etoolbox} % if clauses
\usepackage{pgf, pgfmath, pgfplots, pgfplotstable} % pgf functions
%
%-------------------------------------------------------------------------------
% Structure path prefix
% Note: The prefix is used to abbreviate long structure paths (variable names)
%
% Define default value of structure path prefix
% Do not change that value unless you know what you are doing!
\gdef\OTpfx{oct2tex}
%
% Set structure path prefix to a user defined value
%   Parameter #1: user defined prefix (string without whitespace)
%   Usage: \OTsetpfx{oct2tex.my.pre.fix}
\newcommand{\OTsetpfx}[1]{\ifstrempty{#1}{\gdef\OTpfx{oct2tex}}{\gdef\OTpfx{#1}}}
%
% Reset structure path prefix to default value
\newcommand{\OTresetpfx}{\gdef\OTpfx{oct2tex}}
%
%-------------------------------------------------------------------------------
% Use serialized content from data structures in the document
%
% Use structure variable
%   Parameter #1: variable name (structure path)
%   Usage: \OTuse{my.struct.path.to.content.value}
\newcommand{\OTuse}[1]{\csname \OTpfx.#1\endcsname}
%
% Use structure variable, fixed digit floating point number
%   Parameter #1: variable name (structure path)
%   Parameter #2: number of digits to display
%   Usage: \OTusefixed{my.struct.path.to.content.value}{2}
\newcommand{\OTusefixed}[2]{%
	\pgfkeys{%
		/pgf/number format/.cd,%
		fixed,%
		precision=#2,%
		1000 sep={.}%
	}%
	\pgfmathprintnumber{\OTuse{#1}}%
}
%
% Read tabulated value from structure and store result in the LaTeX command \OTtab
%   Parameter #1: variable name (structure path)
%   Usage: \OTread{my.struct.path.to.table}
\newcommand{\OTread}[1]{\pgfplotstableread[col sep=semicolon,trim cells]{\OTpfx.#1}\OTtab}
%
% Read CSV file and store result in the LaTeX command \OTtabcsv
%   Parameter #1: CSV file name
%   Usage: \OTreadcsv{csv_filename}
\newcommand{\OTreadcsv}[1]{\pgfplotstableread[col sep=semicolon,trim cells]{#1}\OTtabcsv}


% title page settings
\title{Minimal working application example}
\author{Jakob Harden${}^\dag$\\${}^\dag$Graz University of Technology (Graz, Austria)}
\date{August 26${}^{\text{th}}$, 2023}

% begin document body
\begin{document}

	% print title page with abstract
	\maketitle
	
	% print abstract
	\begin{abstract}
		This document contains an application example that clearly shows the use of exported data in \LaTeX{} documents. In particular, how to use and print metadata, ultrasonic signal data and temperature measurements.
	\end{abstract}
	
	% begin with content on a new page
	\newpage

	% import serialized test series metadata
	%% This file was written by Dataset Exporter
%% Dataset Exporter is a script collection conceived, implemented and tested by Jakob Harden (jakob.harden@tugraz.at, Graz University of Technology)
%% It is licenced under the MIT license and has been published under the following URL: https://doi.org/10.3217/9adsn-8dv64
%% To make use of the exported data in your LaTeX document in a convenient way, also include the following script file after the preamble: % TeX commands to conveniently use serialized dataset content
%
% Copyright 2023 Jakob Harden (jakob.harden@tugraz.at, Graz University of Technology, Graz, Austria)
% License: MIT
% This file is part of the PhD thesis of Jakob Harden.
% 
% Permission is hereby granted, free of charge, to any person obtaining a copy of this software and associated 
% documentation files (the “Software”), to deal in the Software without restriction, including without 
% limitation the rights to use, copy, modify, merge, publish, distribute, sublicense, and/or sell copies of 
% the Software, and to permit persons to whom the Software is furnished to do so, subject to the following conditions:
% 
% THE SOFTWARE IS PROVIDED “AS IS”, WITHOUT WARRANTY OF ANY KIND, EXPRESS OR IMPLIED, INCLUDING BUT NOT LIMITED TO 
% THE WARRANTIES OF MERCHANTABILITY, FITNESS FOR A PARTICULAR PURPOSE AND NONINFRINGEMENT. IN NO EVENT SHALL THE 
% AUTHORS OR COPYRIGHT HOLDERS BE LIABLE FOR ANY CLAIM, DAMAGES OR OTHER LIABILITY, WHETHER IN AN ACTION OF CONTRACT, 
% TORT OR OTHERWISE, ARISING FROM, OUT OF OR IN CONNECTION WITH THE SOFTWARE OR THE USE OR OTHER DEALINGS IN THE SOFTWARE.
%
%-------------------------------------------------------------------------------
% Load etoolbox and other required pgf packages
\usepackage{etoolbox} % if clauses
\usepackage{pgf, pgfmath, pgfplots, pgfplotstable} % pgf functions
%
%-------------------------------------------------------------------------------
% Structure path prefix
% Note: The prefix is used to abbreviate long structure paths (variable names)
%
% Define default value of structure path prefix
% Do not change that value unless you know what you are doing!
\gdef\OTpfx{oct2tex}
%
% Set structure path prefix to a user defined value
%   Parameter #1: user defined prefix (string without whitespace)
%   Usage: \OTsetpfx{oct2tex.my.pre.fix}
\newcommand{\OTsetpfx}[1]{\ifstrempty{#1}{\gdef\OTpfx{oct2tex}}{\gdef\OTpfx{#1}}}
%
% Reset structure path prefix to default value
\newcommand{\OTresetpfx}{\gdef\OTpfx{oct2tex}}
%
%-------------------------------------------------------------------------------
% Use serialized content from data structures in the document
%
% Use structure variable
%   Parameter #1: variable name (structure path)
%   Usage: \OTuse{my.struct.path.to.content.value}
\newcommand{\OTuse}[1]{\csname \OTpfx.#1\endcsname}
%
% Use structure variable, fixed digit floating point number
%   Parameter #1: variable name (structure path)
%   Parameter #2: number of digits to display
%   Usage: \OTusefixed{my.struct.path.to.content.value}{2}
\newcommand{\OTusefixed}[2]{%
	\pgfkeys{%
		/pgf/number format/.cd,%
		fixed,%
		precision=#2,%
		1000 sep={.}%
	}%
	\pgfmathprintnumber{\OTuse{#1}}%
}
%
% Read tabulated value from structure and store result in the LaTeX command \OTtab
%   Parameter #1: variable name (structure path)
%   Usage: \OTread{my.struct.path.to.table}
\newcommand{\OTread}[1]{\pgfplotstableread[col sep=semicolon,trim cells]{\OTpfx.#1}\OTtab}
%
% Read CSV file and store result in the LaTeX command \OTtabcsv
%   Parameter #1: CSV file name
%   Usage: \OTreadcsv{csv_filename}
\newcommand{\OTreadcsv}[1]{\pgfplotstableread[col sep=semicolon,trim cells]{#1}\OTtabcsv}

%% The respective TeX script file (oct2texdefs.tex) is stored in the following directory: ./tex/latex/
%% To import this file into your LaTeX document, use the following statement: \input{<filename>}
%%
%% scalar structure
\expandafter\def\csname oct2tex.ts1_wc040_d50_6.meta_ser.obj\endcsname{struct\_metaser}
\expandafter\def\csname oct2tex.ts1_wc040_d50_6.meta_ser.ver\endcsname{1.0}
%% atomic reference element (ARE)
\expandafter\def\csname oct2tex.ts1_wc040_d50_6.meta_ser.r01.obj\endcsname{ARE}
\expandafter\def\csname oct2tex.ts1_wc040_d50_6.meta_ser.r01.ver\endcsname{1.0}
\expandafter\def\csname oct2tex.ts1_wc040_d50_6.meta_ser.r01.t\endcsname{author}
\expandafter\def\csname oct2tex.ts1_wc040_d50_6.meta_ser.r01.d\endcsname{author reference}
\expandafter\def\csname oct2tex.ts1_wc040_d50_6.meta_ser.r01.i\endcsname{1}
\expandafter\def\csname oct2tex.ts1_wc040_d50_6.meta_ser.r01.r\endcsname{aut}
%% atomic reference element (ARE)
\expandafter\def\csname oct2tex.ts1_wc040_d50_6.meta_ser.r02.obj\endcsname{ARE}
\expandafter\def\csname oct2tex.ts1_wc040_d50_6.meta_ser.r02.ver\endcsname{1.0}
\expandafter\def\csname oct2tex.ts1_wc040_d50_6.meta_ser.r02.t\endcsname{license}
\expandafter\def\csname oct2tex.ts1_wc040_d50_6.meta_ser.r02.d\endcsname{license reference}
\expandafter\def\csname oct2tex.ts1_wc040_d50_6.meta_ser.r02.i\endcsname{2}
\expandafter\def\csname oct2tex.ts1_wc040_d50_6.meta_ser.r02.r\endcsname{lic}
%% atomic data element (ADE)
\expandafter\def\csname oct2tex.ts1_wc040_d50_6.meta_ser.d01.obj\endcsname{ADE}
\expandafter\def\csname oct2tex.ts1_wc040_d50_6.meta_ser.d01.ver\endcsname{1.0}
\expandafter\def\csname oct2tex.ts1_wc040_d50_6.meta_ser.d01.t\endcsname{series\_id}
\expandafter\def\csname oct2tex.ts1_wc040_d50_6.meta_ser.d01.d\endcsname{test series id}
\expandafter\def\csname oct2tex.ts1_wc040_d50_6.meta_ser.d01.vt\endcsname{uint}
\expandafter\def\csname oct2tex.ts1_wc040_d50_6.meta_ser.d01.v\endcsname{1}
%% atomic attribute element (AAE)
\expandafter\def\csname oct2tex.ts1_wc040_d50_6.meta_ser.a01.obj\endcsname{AAE}
\expandafter\def\csname oct2tex.ts1_wc040_d50_6.meta_ser.a01.ver\endcsname{1.0}
\expandafter\def\csname oct2tex.ts1_wc040_d50_6.meta_ser.a01.t\endcsname{series\_code}
\expandafter\def\csname oct2tex.ts1_wc040_d50_6.meta_ser.a01.d\endcsname{test series code}
\expandafter\def\csname oct2tex.ts1_wc040_d50_6.meta_ser.a01.v\endcsname{ts1}
%% atomic attribute element (AAE)
\expandafter\def\csname oct2tex.ts1_wc040_d50_6.meta_ser.a02.obj\endcsname{AAE}
\expandafter\def\csname oct2tex.ts1_wc040_d50_6.meta_ser.a02.ver\endcsname{1.0}
\expandafter\def\csname oct2tex.ts1_wc040_d50_6.meta_ser.a02.t\endcsname{series\_name}
\expandafter\def\csname oct2tex.ts1_wc040_d50_6.meta_ser.a02.d\endcsname{test series name}
\expandafter\def\csname oct2tex.ts1_wc040_d50_6.meta_ser.a02.v\endcsname{Test series 1}
%% atomic attribute element (AAE)
\expandafter\def\csname oct2tex.ts1_wc040_d50_6.meta_ser.a03.obj\endcsname{AAE}
\expandafter\def\csname oct2tex.ts1_wc040_d50_6.meta_ser.a03.ver\endcsname{1.0}
\expandafter\def\csname oct2tex.ts1_wc040_d50_6.meta_ser.a03.t\endcsname{description}
\expandafter\def\csname oct2tex.ts1_wc040_d50_6.meta_ser.a03.d\endcsname{test series description}
\expandafter\def\csname oct2tex.ts1_wc040_d50_6.meta_ser.a03.v\endcsname{Ultrasonic pulse transmission tests performed on cement pastes at early stages}
%% atomic attribute element (AAE)
\expandafter\def\csname oct2tex.ts1_wc040_d50_6.meta_ser.a04.obj\endcsname{AAE}
\expandafter\def\csname oct2tex.ts1_wc040_d50_6.meta_ser.a04.ver\endcsname{1.0}
\expandafter\def\csname oct2tex.ts1_wc040_d50_6.meta_ser.a04.t\endcsname{abstract}
\expandafter\def\csname oct2tex.ts1_wc040_d50_6.meta_ser.a04.d\endcsname{test series abstract}
\expandafter\def\csname oct2tex.ts1_wc040_d50_6.meta_ser.a04.v\endcsname{The test series was created to receive information about the material behaviour of cement pastes at early stages. The approach for this test series was to vary the parameters water-cement-ratio and distance-between-actuator-and-sensor. This results in a two-dimensional test parameter grid (water-cement-ratio versus distance-between-actuator-and-sensor). For each point of the parameter grid, tests were performed several times to check the stability of the testing method. The materials tested were blends from ordinary Portland cement and tap water. The test methods used were ultrasonic pulse transmission method with combined compression- and shear wave measurements, gravimetric density tests (fresh paste density, solid specimen density) and hydration temperature tests. All test data and metadata are summarized into datasets using GNU octave's open binary file format.}
%% atomic attribute element (AAE)
\expandafter\def\csname oct2tex.ts1_wc040_d50_6.meta_ser.a05.obj\endcsname{AAE}
\expandafter\def\csname oct2tex.ts1_wc040_d50_6.meta_ser.a05.ver\endcsname{1.0}
\expandafter\def\csname oct2tex.ts1_wc040_d50_6.meta_ser.a05.t\endcsname{context}
\expandafter\def\csname oct2tex.ts1_wc040_d50_6.meta_ser.a05.d\endcsname{test series context}
\expandafter\def\csname oct2tex.ts1_wc040_d50_6.meta_ser.a05.v\endcsname{Test series 1 is part of the PhD thesis of Jakob Harden}
%% atomic attribute element (AAE)
\expandafter\def\csname oct2tex.ts1_wc040_d50_6.meta_ser.a06.obj\endcsname{AAE}
\expandafter\def\csname oct2tex.ts1_wc040_d50_6.meta_ser.a06.ver\endcsname{1.0}
\expandafter\def\csname oct2tex.ts1_wc040_d50_6.meta_ser.a06.t\endcsname{date\_start}
\expandafter\def\csname oct2tex.ts1_wc040_d50_6.meta_ser.a06.d\endcsname{test series start date}
\expandafter\def\csname oct2tex.ts1_wc040_d50_6.meta_ser.a06.v\endcsname{2020-03-10}
%% atomic attribute element (AAE)
\expandafter\def\csname oct2tex.ts1_wc040_d50_6.meta_ser.a07.obj\endcsname{AAE}
\expandafter\def\csname oct2tex.ts1_wc040_d50_6.meta_ser.a07.ver\endcsname{1.0}
\expandafter\def\csname oct2tex.ts1_wc040_d50_6.meta_ser.a07.t\endcsname{date\_end}
\expandafter\def\csname oct2tex.ts1_wc040_d50_6.meta_ser.a07.d\endcsname{test series end date}
\expandafter\def\csname oct2tex.ts1_wc040_d50_6.meta_ser.a07.v\endcsname{2021-03-30}

	
	% import serialized dataset metadata
	%% This file was written by Dataset Exporter
%% Dataset Exporter is a script collection conceived, implemented and tested by Jakob Harden (jakob.harden@tugraz.at, Graz University of Technology)
%% It is licenced under the MIT license and has been published under the following URL: https://doi.org/10.3217/9adsn-8dv64
%% To make use of the exported data in your LaTeX document in a convenient way, also include the following script file after the preamble: % TeX commands to conveniently use serialized dataset content
%
% Copyright 2023 Jakob Harden (jakob.harden@tugraz.at, Graz University of Technology, Graz, Austria)
% License: MIT
% This file is part of the PhD thesis of Jakob Harden.
% 
% Permission is hereby granted, free of charge, to any person obtaining a copy of this software and associated 
% documentation files (the “Software”), to deal in the Software without restriction, including without 
% limitation the rights to use, copy, modify, merge, publish, distribute, sublicense, and/or sell copies of 
% the Software, and to permit persons to whom the Software is furnished to do so, subject to the following conditions:
% 
% THE SOFTWARE IS PROVIDED “AS IS”, WITHOUT WARRANTY OF ANY KIND, EXPRESS OR IMPLIED, INCLUDING BUT NOT LIMITED TO 
% THE WARRANTIES OF MERCHANTABILITY, FITNESS FOR A PARTICULAR PURPOSE AND NONINFRINGEMENT. IN NO EVENT SHALL THE 
% AUTHORS OR COPYRIGHT HOLDERS BE LIABLE FOR ANY CLAIM, DAMAGES OR OTHER LIABILITY, WHETHER IN AN ACTION OF CONTRACT, 
% TORT OR OTHERWISE, ARISING FROM, OUT OF OR IN CONNECTION WITH THE SOFTWARE OR THE USE OR OTHER DEALINGS IN THE SOFTWARE.
%
%-------------------------------------------------------------------------------
% Load etoolbox and other required pgf packages
\usepackage{etoolbox} % if clauses
\usepackage{pgf, pgfmath, pgfplots, pgfplotstable} % pgf functions
%
%-------------------------------------------------------------------------------
% Structure path prefix
% Note: The prefix is used to abbreviate long structure paths (variable names)
%
% Define default value of structure path prefix
% Do not change that value unless you know what you are doing!
\gdef\OTpfx{oct2tex}
%
% Set structure path prefix to a user defined value
%   Parameter #1: user defined prefix (string without whitespace)
%   Usage: \OTsetpfx{oct2tex.my.pre.fix}
\newcommand{\OTsetpfx}[1]{\ifstrempty{#1}{\gdef\OTpfx{oct2tex}}{\gdef\OTpfx{#1}}}
%
% Reset structure path prefix to default value
\newcommand{\OTresetpfx}{\gdef\OTpfx{oct2tex}}
%
%-------------------------------------------------------------------------------
% Use serialized content from data structures in the document
%
% Use structure variable
%   Parameter #1: variable name (structure path)
%   Usage: \OTuse{my.struct.path.to.content.value}
\newcommand{\OTuse}[1]{\csname \OTpfx.#1\endcsname}
%
% Use structure variable, fixed digit floating point number
%   Parameter #1: variable name (structure path)
%   Parameter #2: number of digits to display
%   Usage: \OTusefixed{my.struct.path.to.content.value}{2}
\newcommand{\OTusefixed}[2]{%
	\pgfkeys{%
		/pgf/number format/.cd,%
		fixed,%
		precision=#2,%
		1000 sep={.}%
	}%
	\pgfmathprintnumber{\OTuse{#1}}%
}
%
% Read tabulated value from structure and store result in the LaTeX command \OTtab
%   Parameter #1: variable name (structure path)
%   Usage: \OTread{my.struct.path.to.table}
\newcommand{\OTread}[1]{\pgfplotstableread[col sep=semicolon,trim cells]{\OTpfx.#1}\OTtab}
%
% Read CSV file and store result in the LaTeX command \OTtabcsv
%   Parameter #1: CSV file name
%   Usage: \OTreadcsv{csv_filename}
\newcommand{\OTreadcsv}[1]{\pgfplotstableread[col sep=semicolon,trim cells]{#1}\OTtabcsv}

%% The respective TeX script file (oct2texdefs.tex) is stored in the following directory: ./tex/latex/
%% To import this file into your LaTeX document, use the following statement: \input{<filename>}
%%
%% scalar structure
\expandafter\def\csname oct2tex.ts1_wc040_d50_6.meta_set.obj\endcsname{struct\_metaset}
\expandafter\def\csname oct2tex.ts1_wc040_d50_6.meta_set.ver\endcsname{1.0}
%% atomic reference element (ARE)
\expandafter\def\csname oct2tex.ts1_wc040_d50_6.meta_set.r01.obj\endcsname{ARE}
\expandafter\def\csname oct2tex.ts1_wc040_d50_6.meta_set.r01.ver\endcsname{1.0}
\expandafter\def\csname oct2tex.ts1_wc040_d50_6.meta_set.r01.t\endcsname{author}
\expandafter\def\csname oct2tex.ts1_wc040_d50_6.meta_set.r01.d\endcsname{author reference}
\expandafter\def\csname oct2tex.ts1_wc040_d50_6.meta_set.r01.i\endcsname{1}
\expandafter\def\csname oct2tex.ts1_wc040_d50_6.meta_set.r01.r\endcsname{aut}
%% atomic reference element (ARE)
\expandafter\def\csname oct2tex.ts1_wc040_d50_6.meta_set.r02.obj\endcsname{ARE}
\expandafter\def\csname oct2tex.ts1_wc040_d50_6.meta_set.r02.ver\endcsname{1.0}
\expandafter\def\csname oct2tex.ts1_wc040_d50_6.meta_set.r02.t\endcsname{series}
\expandafter\def\csname oct2tex.ts1_wc040_d50_6.meta_set.r02.d\endcsname{test series reference}
\expandafter\def\csname oct2tex.ts1_wc040_d50_6.meta_set.r02.i\endcsname{1}
\expandafter\def\csname oct2tex.ts1_wc040_d50_6.meta_set.r02.r\endcsname{meta\_ser}
%% atomic reference element (ARE)
\expandafter\def\csname oct2tex.ts1_wc040_d50_6.meta_set.r03.obj\endcsname{ARE}
\expandafter\def\csname oct2tex.ts1_wc040_d50_6.meta_set.r03.ver\endcsname{1.0}
\expandafter\def\csname oct2tex.ts1_wc040_d50_6.meta_set.r03.t\endcsname{location}
\expandafter\def\csname oct2tex.ts1_wc040_d50_6.meta_set.r03.d\endcsname{location reference}
\expandafter\def\csname oct2tex.ts1_wc040_d50_6.meta_set.r03.i\endcsname{1}
\expandafter\def\csname oct2tex.ts1_wc040_d50_6.meta_set.r03.r\endcsname{loc}
%% atomic reference element (ARE)
\expandafter\def\csname oct2tex.ts1_wc040_d50_6.meta_set.r04.obj\endcsname{ARE}
\expandafter\def\csname oct2tex.ts1_wc040_d50_6.meta_set.r04.ver\endcsname{1.0}
\expandafter\def\csname oct2tex.ts1_wc040_d50_6.meta_set.r04.t\endcsname{license}
\expandafter\def\csname oct2tex.ts1_wc040_d50_6.meta_set.r04.d\endcsname{license reference}
\expandafter\def\csname oct2tex.ts1_wc040_d50_6.meta_set.r04.i\endcsname{2}
\expandafter\def\csname oct2tex.ts1_wc040_d50_6.meta_set.r04.r\endcsname{lic}
%% atomic data element (ADE)
\expandafter\def\csname oct2tex.ts1_wc040_d50_6.meta_set.d01.obj\endcsname{ADE}
\expandafter\def\csname oct2tex.ts1_wc040_d50_6.meta_set.d01.ver\endcsname{1.0}
\expandafter\def\csname oct2tex.ts1_wc040_d50_6.meta_set.d01.t\endcsname{dataset\_id}
\expandafter\def\csname oct2tex.ts1_wc040_d50_6.meta_set.d01.d\endcsname{data set id}
\expandafter\def\csname oct2tex.ts1_wc040_d50_6.meta_set.d01.vt\endcsname{uint}
\expandafter\def\csname oct2tex.ts1_wc040_d50_6.meta_set.d01.v\endcsname{84}
%% atomic attribute element (AAE)
\expandafter\def\csname oct2tex.ts1_wc040_d50_6.meta_set.a01.obj\endcsname{AAE}
\expandafter\def\csname oct2tex.ts1_wc040_d50_6.meta_set.a01.ver\endcsname{1.0}
\expandafter\def\csname oct2tex.ts1_wc040_d50_6.meta_set.a01.t\endcsname{dataset\_code}
\expandafter\def\csname oct2tex.ts1_wc040_d50_6.meta_set.a01.d\endcsname{data set code}
\expandafter\def\csname oct2tex.ts1_wc040_d50_6.meta_set.a01.v\endcsname{ts1\_wc040\_d50\_6}
%% atomic attribute element (AAE)
\expandafter\def\csname oct2tex.ts1_wc040_d50_6.meta_set.a02.obj\endcsname{AAE}
\expandafter\def\csname oct2tex.ts1_wc040_d50_6.meta_set.a02.ver\endcsname{1.0}
\expandafter\def\csname oct2tex.ts1_wc040_d50_6.meta_set.a02.t\endcsname{dataset\_name}
\expandafter\def\csname oct2tex.ts1_wc040_d50_6.meta_set.a02.d\endcsname{data set name}
\expandafter\def\csname oct2tex.ts1_wc040_d50_6.meta_set.a02.v\endcsname{data set ts1\_wc040\_d50\_6}
%% atomic attribute element (AAE)
\expandafter\def\csname oct2tex.ts1_wc040_d50_6.meta_set.a03.obj\endcsname{AAE}
\expandafter\def\csname oct2tex.ts1_wc040_d50_6.meta_set.a03.ver\endcsname{1.0}
\expandafter\def\csname oct2tex.ts1_wc040_d50_6.meta_set.a03.t\endcsname{description}
\expandafter\def\csname oct2tex.ts1_wc040_d50_6.meta_set.a03.d\endcsname{description, general}
\expandafter\def\csname oct2tex.ts1_wc040_d50_6.meta_set.a03.v\endcsname{The data set was compiled with GNU octave 6.2.0 using the built in binary file format (*.oct).}
%% atomic attribute element (AAE)
\expandafter\def\csname oct2tex.ts1_wc040_d50_6.meta_set.a04.obj\endcsname{AAE}
\expandafter\def\csname oct2tex.ts1_wc040_d50_6.meta_set.a04.ver\endcsname{1.0}
\expandafter\def\csname oct2tex.ts1_wc040_d50_6.meta_set.a04.t\endcsname{description\_abstract}
\expandafter\def\csname oct2tex.ts1_wc040_d50_6.meta_set.a04.d\endcsname{description, abstract}
\begin{filecontents}[overwrite]{oct2tex.ts1_wc040_d50_6.meta_set.a04.v}
idx ; val
1 ; Data set parameters:
2 ; w/c-ratio = 0.40
3 ; distance = 50mm
4 ; pass = 6
\end{filecontents}
%% atomic attribute element (AAE)
\expandafter\def\csname oct2tex.ts1_wc040_d50_6.meta_set.a05.obj\endcsname{AAE}
\expandafter\def\csname oct2tex.ts1_wc040_d50_6.meta_set.a05.ver\endcsname{1.0}
\expandafter\def\csname oct2tex.ts1_wc040_d50_6.meta_set.a05.t\endcsname{description\_methods}
\expandafter\def\csname oct2tex.ts1_wc040_d50_6.meta_set.a05.d\endcsname{description, methods}
\begin{filecontents}[overwrite]{oct2tex.ts1_wc040_d50_6.meta_set.a05.v}
idx ; val
1 ; Ultrasonic pulse transmission tests (combined compression- and shear wave measurements)
2 ; Density tests (gravimetric)
3 ; Distance measurements
4 ; Temperature tests (thermocouples, temperature logger)
\end{filecontents}
%% atomic attribute element (AAE)
\expandafter\def\csname oct2tex.ts1_wc040_d50_6.meta_set.a06.obj\endcsname{AAE}
\expandafter\def\csname oct2tex.ts1_wc040_d50_6.meta_set.a06.ver\endcsname{1.0}
\expandafter\def\csname oct2tex.ts1_wc040_d50_6.meta_set.a06.t\endcsname{description\_tableofcontents}
\expandafter\def\csname oct2tex.ts1_wc040_d50_6.meta_set.a06.d\endcsname{description, tableofcontents}
\begin{filecontents}[overwrite]{oct2tex.ts1_wc040_d50_6.meta_set.a06.v}
idx ; val
1 ; Data set \ldots dataset
2 ; (1) Test series metadata \ldots dataset.meta\_ser
3 ; (2) Data set metadata \ldots dataset.meta\_set
4 ; (3) Location information \ldots dataset.loc
5 ; (4) License information \ldots dataset.lic
6 ; (5) Author information \ldots dataset.aut
7 ; (6) Device information \ldots dataset.dev
8 ; (7) Material information \ldots dataset.mat
9 ; (8) Mixture recipe information \ldots dataset.rec
10 ; (9) Mixture information \ldots dataset.mix
11 ; (10) Specimen information \ldots dataset.spm
12 ; (11) Test collection \ldots dataset.tst
13 ; (11.1) Fresh paste density test \ldots dataset.tst.s01
14 ; (11.2) Solid specimen density test 1 (gravimetric, immersion weighing) \ldots dataset.tst.s02
15 ; (11.3) Solid specimen density test 1 (gravimetric, immersion weighing) \ldots dataset.tst.s03
16 ; (11.4) Ultrasonic measurement distance test 1 (caliper, spacer) \ldots dataset.tst.s04
17 ; (11.5) Ultrasonic measurement distance test 1 (caliper, spacer) \ldots dataset.tst.s05
18 ; (11.6) Ultrasonic pulse transmission test \ldots dataset.tst.s06
19 ; (11.7) Ultrasonic pulse transmission test \ldots dataset.tst.s07
20 ; (11.8) Specimen temperature test \ldots dataset.tst.s08
\end{filecontents}
%% atomic attribute element (AAE)
\expandafter\def\csname oct2tex.ts1_wc040_d50_6.meta_set.a07.obj\endcsname{AAE}
\expandafter\def\csname oct2tex.ts1_wc040_d50_6.meta_set.a07.ver\endcsname{1.0}
\expandafter\def\csname oct2tex.ts1_wc040_d50_6.meta_set.a07.t\endcsname{created\_by}
\expandafter\def\csname oct2tex.ts1_wc040_d50_6.meta_set.a07.d\endcsname{data set creator name}
\expandafter\def\csname oct2tex.ts1_wc040_d50_6.meta_set.a07.v\endcsname{Harden, Jakob (jakob.harden@tugraz.at)}
%% atomic attribute element (AAE)
\expandafter\def\csname oct2tex.ts1_wc040_d50_6.meta_set.a08.obj\endcsname{AAE}
\expandafter\def\csname oct2tex.ts1_wc040_d50_6.meta_set.a08.ver\endcsname{1.0}
\expandafter\def\csname oct2tex.ts1_wc040_d50_6.meta_set.a08.t\endcsname{collected\_by}
\expandafter\def\csname oct2tex.ts1_wc040_d50_6.meta_set.a08.d\endcsname{data set collector name}
\expandafter\def\csname oct2tex.ts1_wc040_d50_6.meta_set.a08.v\endcsname{Harden, Jakob (jakob.harden@tugraz.at)}
%% atomic attribute element (AAE)
\expandafter\def\csname oct2tex.ts1_wc040_d50_6.meta_set.a09.obj\endcsname{AAE}
\expandafter\def\csname oct2tex.ts1_wc040_d50_6.meta_set.a09.ver\endcsname{1.0}
\expandafter\def\csname oct2tex.ts1_wc040_d50_6.meta_set.a09.t\endcsname{copyrighted\_by}
\expandafter\def\csname oct2tex.ts1_wc040_d50_6.meta_set.a09.d\endcsname{data set copyrighter name}
\expandafter\def\csname oct2tex.ts1_wc040_d50_6.meta_set.a09.v\endcsname{Harden, Jakob (jakob.harden@tugraz.at)}
%% atomic attribute element (AAE)
\expandafter\def\csname oct2tex.ts1_wc040_d50_6.meta_set.a10.obj\endcsname{AAE}
\expandafter\def\csname oct2tex.ts1_wc040_d50_6.meta_set.a10.ver\endcsname{1.0}
\expandafter\def\csname oct2tex.ts1_wc040_d50_6.meta_set.a10.t\endcsname{date\_created}
\expandafter\def\csname oct2tex.ts1_wc040_d50_6.meta_set.a10.d\endcsname{date created}
\expandafter\def\csname oct2tex.ts1_wc040_d50_6.meta_set.a10.v\endcsname{2021-03-15}
%% atomic attribute element (AAE)
\expandafter\def\csname oct2tex.ts1_wc040_d50_6.meta_set.a11.obj\endcsname{AAE}
\expandafter\def\csname oct2tex.ts1_wc040_d50_6.meta_set.a11.ver\endcsname{1.0}
\expandafter\def\csname oct2tex.ts1_wc040_d50_6.meta_set.a11.t\endcsname{date\_collected}
\expandafter\def\csname oct2tex.ts1_wc040_d50_6.meta_set.a11.d\endcsname{date collected}
\expandafter\def\csname oct2tex.ts1_wc040_d50_6.meta_set.a11.v\endcsname{2023-01-03}
%% atomic attribute element (AAE)
\expandafter\def\csname oct2tex.ts1_wc040_d50_6.meta_set.a12.obj\endcsname{AAE}
\expandafter\def\csname oct2tex.ts1_wc040_d50_6.meta_set.a12.ver\endcsname{1.0}
\expandafter\def\csname oct2tex.ts1_wc040_d50_6.meta_set.a12.t\endcsname{date\_copyrighted}
\expandafter\def\csname oct2tex.ts1_wc040_d50_6.meta_set.a12.d\endcsname{date copyrighted}
\expandafter\def\csname oct2tex.ts1_wc040_d50_6.meta_set.a12.v\endcsname{2023-01-03}
%% atomic attribute element (AAE)
\expandafter\def\csname oct2tex.ts1_wc040_d50_6.meta_set.a13.obj\endcsname{AAE}
\expandafter\def\csname oct2tex.ts1_wc040_d50_6.meta_set.a13.ver\endcsname{1.0}
\expandafter\def\csname oct2tex.ts1_wc040_d50_6.meta_set.a13.t\endcsname{size}
\expandafter\def\csname oct2tex.ts1_wc040_d50_6.meta_set.a13.d\endcsname{data set size, number of files}
\expandafter\def\csname oct2tex.ts1_wc040_d50_6.meta_set.a13.v\endcsname{1 file}
%% atomic attribute element (AAE)
\expandafter\def\csname oct2tex.ts1_wc040_d50_6.meta_set.a14.obj\endcsname{AAE}
\expandafter\def\csname oct2tex.ts1_wc040_d50_6.meta_set.a14.ver\endcsname{1.0}
\expandafter\def\csname oct2tex.ts1_wc040_d50_6.meta_set.a14.t\endcsname{format}
\expandafter\def\csname oct2tex.ts1_wc040_d50_6.meta_set.a14.d\endcsname{data set file format}
\expandafter\def\csname oct2tex.ts1_wc040_d50_6.meta_set.a14.v\endcsname{application/octet-stream}
%% atomic attribute element (AAE)
\expandafter\def\csname oct2tex.ts1_wc040_d50_6.meta_set.a15.obj\endcsname{AAE}
\expandafter\def\csname oct2tex.ts1_wc040_d50_6.meta_set.a15.ver\endcsname{1.0}
\expandafter\def\csname oct2tex.ts1_wc040_d50_6.meta_set.a15.t\endcsname{version}
\expandafter\def\csname oct2tex.ts1_wc040_d50_6.meta_set.a15.d\endcsname{data set version}
\expandafter\def\csname oct2tex.ts1_wc040_d50_6.meta_set.a15.v\endcsname{1.0}
%% atomic attribute element (AAE)
\expandafter\def\csname oct2tex.ts1_wc040_d50_6.meta_set.a16.obj\endcsname{AAE}
\expandafter\def\csname oct2tex.ts1_wc040_d50_6.meta_set.a16.ver\endcsname{1.0}
\expandafter\def\csname oct2tex.ts1_wc040_d50_6.meta_set.a16.t\endcsname{context}
\expandafter\def\csname oct2tex.ts1_wc040_d50_6.meta_set.a16.d\endcsname{data set context}
\expandafter\def\csname oct2tex.ts1_wc040_d50_6.meta_set.a16.v\endcsname{This data set is part of the PhD thesis of Jakob Harden.}
%% atomic attribute element (AAE)
\expandafter\def\csname oct2tex.ts1_wc040_d50_6.meta_set.a17.obj\endcsname{AAE}
\expandafter\def\csname oct2tex.ts1_wc040_d50_6.meta_set.a17.ver\endcsname{1.0}
\expandafter\def\csname oct2tex.ts1_wc040_d50_6.meta_set.a17.t\endcsname{rawdata\_directory}
\expandafter\def\csname oct2tex.ts1_wc040_d50_6.meta_set.a17.d\endcsname{data set rawdata directory}
\expandafter\def\csname oct2tex.ts1_wc040_d50_6.meta_set.a17.v\endcsname{ts1\_wc040\_d50\_6}
%% atomic attribute element (AAE)
\expandafter\def\csname oct2tex.ts1_wc040_d50_6.meta_set.a18.obj\endcsname{AAE}
\expandafter\def\csname oct2tex.ts1_wc040_d50_6.meta_set.a18.ver\endcsname{1.0}
\expandafter\def\csname oct2tex.ts1_wc040_d50_6.meta_set.a18.t\endcsname{rawdata\_archive}
\expandafter\def\csname oct2tex.ts1_wc040_d50_6.meta_set.a18.d\endcsname{data set rawdata archive}
\expandafter\def\csname oct2tex.ts1_wc040_d50_6.meta_set.a18.v\endcsname{ts1\_wc040\_d50\_6.zip}

	
	% import serialized environment/ambient temperature test data
	%% This file was written by Dataset Exporter
%% Dataset Exporter is a script collection conceived, implemented and tested by Jakob Harden (jakob.harden@tugraz.at, Graz University of Technology)
%% It is licenced under the MIT license and has been published under the following URL: https://doi.org/10.3217/9adsn-8dv64
%% To make use of the exported data in your LaTeX document in a convenient way, also include the following script file after the preamble: % TeX commands to conveniently use serialized dataset content
%
% Copyright 2023 Jakob Harden (jakob.harden@tugraz.at, Graz University of Technology, Graz, Austria)
% License: MIT
% This file is part of the PhD thesis of Jakob Harden.
% 
% Permission is hereby granted, free of charge, to any person obtaining a copy of this software and associated 
% documentation files (the “Software”), to deal in the Software without restriction, including without 
% limitation the rights to use, copy, modify, merge, publish, distribute, sublicense, and/or sell copies of 
% the Software, and to permit persons to whom the Software is furnished to do so, subject to the following conditions:
% 
% THE SOFTWARE IS PROVIDED “AS IS”, WITHOUT WARRANTY OF ANY KIND, EXPRESS OR IMPLIED, INCLUDING BUT NOT LIMITED TO 
% THE WARRANTIES OF MERCHANTABILITY, FITNESS FOR A PARTICULAR PURPOSE AND NONINFRINGEMENT. IN NO EVENT SHALL THE 
% AUTHORS OR COPYRIGHT HOLDERS BE LIABLE FOR ANY CLAIM, DAMAGES OR OTHER LIABILITY, WHETHER IN AN ACTION OF CONTRACT, 
% TORT OR OTHERWISE, ARISING FROM, OUT OF OR IN CONNECTION WITH THE SOFTWARE OR THE USE OR OTHER DEALINGS IN THE SOFTWARE.
%
%-------------------------------------------------------------------------------
% Load etoolbox and other required pgf packages
\usepackage{etoolbox} % if clauses
\usepackage{pgf, pgfmath, pgfplots, pgfplotstable} % pgf functions
%
%-------------------------------------------------------------------------------
% Structure path prefix
% Note: The prefix is used to abbreviate long structure paths (variable names)
%
% Define default value of structure path prefix
% Do not change that value unless you know what you are doing!
\gdef\OTpfx{oct2tex}
%
% Set structure path prefix to a user defined value
%   Parameter #1: user defined prefix (string without whitespace)
%   Usage: \OTsetpfx{oct2tex.my.pre.fix}
\newcommand{\OTsetpfx}[1]{\ifstrempty{#1}{\gdef\OTpfx{oct2tex}}{\gdef\OTpfx{#1}}}
%
% Reset structure path prefix to default value
\newcommand{\OTresetpfx}{\gdef\OTpfx{oct2tex}}
%
%-------------------------------------------------------------------------------
% Use serialized content from data structures in the document
%
% Use structure variable
%   Parameter #1: variable name (structure path)
%   Usage: \OTuse{my.struct.path.to.content.value}
\newcommand{\OTuse}[1]{\csname \OTpfx.#1\endcsname}
%
% Use structure variable, fixed digit floating point number
%   Parameter #1: variable name (structure path)
%   Parameter #2: number of digits to display
%   Usage: \OTusefixed{my.struct.path.to.content.value}{2}
\newcommand{\OTusefixed}[2]{%
	\pgfkeys{%
		/pgf/number format/.cd,%
		fixed,%
		precision=#2,%
		1000 sep={.}%
	}%
	\pgfmathprintnumber{\OTuse{#1}}%
}
%
% Read tabulated value from structure and store result in the LaTeX command \OTtab
%   Parameter #1: variable name (structure path)
%   Usage: \OTread{my.struct.path.to.table}
\newcommand{\OTread}[1]{\pgfplotstableread[col sep=semicolon,trim cells]{\OTpfx.#1}\OTtab}
%
% Read CSV file and store result in the LaTeX command \OTtabcsv
%   Parameter #1: CSV file name
%   Usage: \OTreadcsv{csv_filename}
\newcommand{\OTreadcsv}[1]{\pgfplotstableread[col sep=semicolon,trim cells]{#1}\OTtabcsv}

%% The respective TeX script file (oct2texdefs.tex) is stored in the following directory: ./tex/latex/
%% To import this file into your LaTeX document, use the following statement: \input{<filename>}
%%
%% scalar structure
\expandafter\def\csname oct2tex.ts1_wc040_d50_6.tst.env.obj\endcsname{struct\_test\_env1}
\expandafter\def\csname oct2tex.ts1_wc040_d50_6.tst.env.ver\endcsname{1.0}
%% atomic reference element (ARE)
\expandafter\def\csname oct2tex.ts1_wc040_d50_6.tst.env.r01.obj\endcsname{ARE}
\expandafter\def\csname oct2tex.ts1_wc040_d50_6.tst.env.r01.ver\endcsname{1.0}
\expandafter\def\csname oct2tex.ts1_wc040_d50_6.tst.env.r01.t\endcsname{author}
\expandafter\def\csname oct2tex.ts1_wc040_d50_6.tst.env.r01.d\endcsname{author reference}
\expandafter\def\csname oct2tex.ts1_wc040_d50_6.tst.env.r01.i\endcsname{1}
\expandafter\def\csname oct2tex.ts1_wc040_d50_6.tst.env.r01.r\endcsname{aut}
%% atomic reference element (ARE)
\expandafter\def\csname oct2tex.ts1_wc040_d50_6.tst.env.r02.obj\endcsname{ARE}
\expandafter\def\csname oct2tex.ts1_wc040_d50_6.tst.env.r02.ver\endcsname{1.0}
\expandafter\def\csname oct2tex.ts1_wc040_d50_6.tst.env.r02.t\endcsname{device}
\expandafter\def\csname oct2tex.ts1_wc040_d50_6.tst.env.r02.d\endcsname{device reference}
\expandafter\def\csname oct2tex.ts1_wc040_d50_6.tst.env.r02.i\endcsname{33}
\expandafter\def\csname oct2tex.ts1_wc040_d50_6.tst.env.r02.r\endcsname{dev}
%% atomic reference element (ARE)
\expandafter\def\csname oct2tex.ts1_wc040_d50_6.tst.env.r03.obj\endcsname{ARE}
\expandafter\def\csname oct2tex.ts1_wc040_d50_6.tst.env.r03.ver\endcsname{1.0}
\expandafter\def\csname oct2tex.ts1_wc040_d50_6.tst.env.r03.t\endcsname{location}
\expandafter\def\csname oct2tex.ts1_wc040_d50_6.tst.env.r03.d\endcsname{location reference}
\expandafter\def\csname oct2tex.ts1_wc040_d50_6.tst.env.r03.i\endcsname{1}
\expandafter\def\csname oct2tex.ts1_wc040_d50_6.tst.env.r03.r\endcsname{loc}
%% atomic data element (ADE)
\expandafter\def\csname oct2tex.ts1_wc040_d50_6.tst.env.d01.obj\endcsname{ADE}
\expandafter\def\csname oct2tex.ts1_wc040_d50_6.tst.env.d01.ver\endcsname{1.0}
\expandafter\def\csname oct2tex.ts1_wc040_d50_6.tst.env.d01.t\endcsname{datetime}
\expandafter\def\csname oct2tex.ts1_wc040_d50_6.tst.env.d01.d\endcsname{date and time, seconds since epoch (UTC)}
\expandafter\def\csname oct2tex.ts1_wc040_d50_6.tst.env.d01.u\endcsname{\ensuremath{utc}}
\expandafter\def\csname oct2tex.ts1_wc040_d50_6.tst.env.d01.vt\endcsname{dbl}
\expandafter\def\csname oct2tex.ts1_wc040_d50_6.tst.env.d01.v\endcsname{0.0000000000000000}
%% atomic data element (ADE)
\expandafter\def\csname oct2tex.ts1_wc040_d50_6.tst.env.d02.obj\endcsname{ADE}
\expandafter\def\csname oct2tex.ts1_wc040_d50_6.tst.env.d02.ver\endcsname{1.0}
\expandafter\def\csname oct2tex.ts1_wc040_d50_6.tst.env.d02.t\endcsname{temperature}
\expandafter\def\csname oct2tex.ts1_wc040_d50_6.tst.env.d02.d\endcsname{environment temperature at test start}
\expandafter\def\csname oct2tex.ts1_wc040_d50_6.tst.env.d02.u\endcsname{\ensuremath{°C}}
\expandafter\def\csname oct2tex.ts1_wc040_d50_6.tst.env.d02.vt\endcsname{dbl}
\expandafter\def\csname oct2tex.ts1_wc040_d50_6.tst.env.d02.v\endcsname{19.6999999999999993}
%% atomic attribute element (AAE)
\expandafter\def\csname oct2tex.ts1_wc040_d50_6.tst.env.a01.obj\endcsname{AAE}
\expandafter\def\csname oct2tex.ts1_wc040_d50_6.tst.env.a01.ver\endcsname{1.0}
\expandafter\def\csname oct2tex.ts1_wc040_d50_6.tst.env.a01.t\endcsname{testname}
\expandafter\def\csname oct2tex.ts1_wc040_d50_6.tst.env.a01.d\endcsname{test name}
\expandafter\def\csname oct2tex.ts1_wc040_d50_6.tst.env.a01.v\endcsname{Environment temperature (room air)}
%% atomic attribute element (AAE)
\expandafter\def\csname oct2tex.ts1_wc040_d50_6.tst.env.a02.obj\endcsname{AAE}
\expandafter\def\csname oct2tex.ts1_wc040_d50_6.tst.env.a02.ver\endcsname{1.0}
\expandafter\def\csname oct2tex.ts1_wc040_d50_6.tst.env.a02.t\endcsname{operator}
\expandafter\def\csname oct2tex.ts1_wc040_d50_6.tst.env.a02.d\endcsname{operator name}
\expandafter\def\csname oct2tex.ts1_wc040_d50_6.tst.env.a02.v\endcsname{Jakob Harden (jakob.harden@tugraz.at)}
%% atomic attribute element (AAE)
\expandafter\def\csname oct2tex.ts1_wc040_d50_6.tst.env.a03.obj\endcsname{AAE}
\expandafter\def\csname oct2tex.ts1_wc040_d50_6.tst.env.a03.ver\endcsname{1.0}
\expandafter\def\csname oct2tex.ts1_wc040_d50_6.tst.env.a03.t\endcsname{procedure}
\expandafter\def\csname oct2tex.ts1_wc040_d50_6.tst.env.a03.d\endcsname{procedure description}
\begin{filecontents}[overwrite]{oct2tex.ts1_wc040_d50_6.tst.env.a03.v}
idx ; val
1 ; (1) Read temperature from display of temperature logger mounted on the wall
\end{filecontents}
%% atomic attribute element (AAE)
\expandafter\def\csname oct2tex.ts1_wc040_d50_6.tst.env.a04.obj\endcsname{AAE}
\expandafter\def\csname oct2tex.ts1_wc040_d50_6.tst.env.a04.ver\endcsname{1.0}
\expandafter\def\csname oct2tex.ts1_wc040_d50_6.tst.env.a04.t\endcsname{calculation}
\expandafter\def\csname oct2tex.ts1_wc040_d50_6.tst.env.a04.d\endcsname{calculation description, formula}
\begin{filecontents}[overwrite]{oct2tex.ts1_wc040_d50_6.tst.env.a04.v}
idx ; val
1 ; Not required
\end{filecontents}
%% atomic attribute element (AAE)
\expandafter\def\csname oct2tex.ts1_wc040_d50_6.tst.env.a05.obj\endcsname{AAE}
\expandafter\def\csname oct2tex.ts1_wc040_d50_6.tst.env.a05.ver\endcsname{1.0}
\expandafter\def\csname oct2tex.ts1_wc040_d50_6.tst.env.a05.t\endcsname{description}
\expandafter\def\csname oct2tex.ts1_wc040_d50_6.tst.env.a05.d\endcsname{general description}
\expandafter\def\csname oct2tex.ts1_wc040_d50_6.tst.env.a05.v\endcsname{No additional information available}

	
	% set default variable name prefix
	% this makes the variable names which are the data structure paths much shorter
	\OTsetpfx{oct2tex.ts1_wc040_d50_6}
	
	% use test series metadata
	The dataset used in this example is included in \textcolor{red}{\OTuse{meta_ser.a02.v}}\cite{datacem}.
	% use dataset metadata
	The code of the dataset is \textcolor{red}{\OTuse{meta_set.a01.v}}.
	% use test data
	The ambient temperature before the start of the ultrasonic pulse transmission test was: \textcolor{red}{\OTusefixed{tst.env.d02.v}{2} \OTuse{tst.env.d02.u}}
	
	\paragraph{Ultrasonic signal data:} The following graph shows the signal response of a compression wave and a shear wave at a maturity of 24 hours.
	
	% plot signal data
	\begin{figure}[ht]
		\centering
		\begin{tikzpicture}
			\begin{axis}[%
				xlabel={Time [$\mu sec$]},%
				ylabel={Magnitude [V]},%
				width=12cm, height=5.5cm,%
				grid]
				% read signal data of compression wave (table is stored in variable \OTtabcsv)
				\OTreadcsv{ts1_wc040_d50_6_sig_c.csv}
				% x expr: scale unit, seconds to micro-seconds
				% plot compression wave
				\addplot[thick,red] table [x expr={\thisrowno{1}*1000000}, y index={2}] {\OTtabcsv};
				% read signal data of shear wave (table is stored in variable \OTtabcsv)
				\OTreadcsv{ts1_wc040_d50_6_sig_s.csv}
				% plot shear wave
				\addplot[thick,blue] table [x expr={\thisrowno{1}*1000000}, y index={2}] {\OTtabcsv};
				\legend{C, S};
			\end{axis}
		\end{tikzpicture}
		\caption{Compression (C) and shear (S) wave at a specimen maturity of 24 hours.}
	\end{figure}
	
	\paragraph{Specimen temperature data:} The following graph shows the evolution of the specimen temperature within the first 24 hours of the hydration process.
	
	% plot signal data
	\begin{figure}[ht]
		\centering
		\begin{tikzpicture}
			\begin{axis}[%
				xlabel={Maturity [min]},%
				ylabel={Temperature [${}^\circ C$]},%
				width=12cm, height=5.5cm,%
				grid]
				% setup tick labels
				\pgfkeys{%
					/pgf/number format/.cd,%
					fixed,%
					precision=1,%
					1000 sep={}%
				}
				% read temperature data (table is stored in variable \OTtabcsv)
				\OTreadcsv{ts1_wc040_d50_6_tem_t1-t4.csv}
				% x expr: scale unit, seconds to minutes
				% plot channel 1
				\addplot[thick,red] table [x expr={\thisrowno{1}/60}, y index={2}] {\OTtabcsv};
				% plot channel 2
				\addplot[thick,green] table [x expr={\thisrowno{1}/60}, y index={3}] {\OTtabcsv};
				% plot channel 3
				\addplot[thick,blue] table [x expr={\thisrowno{1}/60}, y index={4}] {\OTtabcsv};
				% plot channel 4
				\addplot[thick,black] table [x expr={\thisrowno{1}/60}, y index={5}] {\OTtabcsv};
				\legend{$T1$, $T2$, $T3$, $T4$};
			\end{axis}
		\end{tikzpicture}
		\caption{Specimen temperatures and ambient temperature. Legend: $T1$~\ldots~channel 1 (specimen); $T2$~\ldots~channel 2 (specimen); $T3$~\ldots~channel 3 (specimen); $T4$~\ldots~channel 4 (ambient air)}
	\end{figure}
	
	% Print references
	\printbibliography
	
% end document body
\end{document}
